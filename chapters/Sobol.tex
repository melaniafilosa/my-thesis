\chapter{Implementation of Sobol' sequences}\label{app:Sobol}
\section{Van der Corput sequences}
 In the following we show a particular construction of a low-discrepancy sequence for $d=1$ that was introduced the first time by Van der Corput in 1935.  
This kind of sequences, called \textit{van der Corput} sequences, are particular interesting not only because they give an intuition of how to construct low discrepancy sequences but also because many other kind of sequences in higher dimensions are based on this one-dimensional case. Before introducing these sequences we need to give the concept of radical inverse function. Let $\const{b}\geq 2$ be an integer base. Any natural number $\variabile{n}\in \mathbb{N}_0$ can be decomposed in base $\const{b}$ as follows:
\begin{equation}
\variabile{n} = \sum_{\variabile{i}=0}^\infty \variabile{d}_{\variabile{i}}\const{b}^{\variabile{i}}
\end{equation}
where $\variabile{d}_{\variabile{i}} \in \{0, 1, \cdots, \const{b}-1\}$ are the digit numbers.
The radical inverse function $\phi_{\const{b}}:\mathbb{N}_0\mapsto [0,1)$ in base $\const{b}$ is defined as:
\begin{equation}
\phi_{\const{b}}(\variabile{n}) = \sum_{\variabile{i}=1}^{\infty}\frac{\variabile{d}_{\variabile{i}-1}}{\const{b}^{\variabile{i}}}.
\end{equation}
As an example we provide in the following the radical inverse function $\phi_{\const{b}}(5)$ in base $\const{b} = 2$. 
The digit expansion in base $\const{b}$ of $\variabile{n}=5$ is:
\begin{equation}
5 = 1\cdot 2^0+1\cdot 2^2.
\end{equation}
Therefore, $\variabile{d}_0 = 1, \variabile{d}_1 = 0$ and $\variabile{d}_2 = 1$. 
The radical inverse function $\phi_2(5)$ is:
\begin{equation}
\phi_2 (5) = \frac{1}{2}+\frac{1}{8} = \frac{5}{8}.
\end{equation}
\begin{definition}
The Van der Corput sequence in base $\const{b}$ is defined as $\{ \phi_{\const{b}}(\variabile{n})\}_{n\in\mathbb{N}_0}$.
\end{definition}
For example, suppose we have the finite sequence of numbers $\variabile{n}\in \{0, 1,\cdots, 8\}$  the corresponding Van der Corput sequence 
$\{ \phi_{\const{b}}(\variabile{n})_{\variabile{n}\in \{0, 1,\cdots, 8\}}$ in base $\variabile{b}=2$ is:
\begin{equation}
\big\{\phi_2(\variabile{n})\big\}_{\variabile{n}\in \{0, 1,\cdots, 8\}} = \Bigg\{0, \frac{1}{2}, \frac{1}{4}, \frac{3}{4}, \frac{1}{8},\frac{5}{8}, \frac{3}{8}, \frac{7}{8}, \frac{1}{16}\Bigg\} \,.
\end{equation}
It can be proved that the Van der Corput sequence in base $\variabile{b}$ is uniformly distributed modulo one, \cite{leobacher2014introduction}. 
The van der Corput sequence has been extended to higher dimensions. 
The most common QMC approach uses Sobol sequence which is one an extended Van der Corput sequence in base $\variabile{b}=2$ to $\variabile{d}\geq2$. 
\section{Sobol' sequences}