%\documentclass{article}

\newcommand{\set}[3]{\emph{#1}_{\textit{#2#3}}}
\newcommand{\map}[3]{\mathrm{#1}_{\textit{#2#3}}}
\newcommand{\setbound}[4]{\emph{#1}_{\textit{#2#3}}^{#4}}
\newcommand{\variabile}[1]{\textit{#1}}
\newcommand{\inversemap}[3]{\mathrm{#1}_{\textit{#2#3}}^{-1}}
\newcommand{\vect}[1]{\textit{\textbf{#1}}}
\newcommand{\point}[1]{\textsf{#1}}
\newcommand{\scalar}[2]{(#1 #2)}
\newcommand{\const}[1]{\textrm{#1}}
\newcommand{\myangle}{\theta}
\newcommand{\mytime}{T}
\newcommand{\mynormal}{$\boldsymbol{\nu}$}
\newcommand{\lineai}{\variabile{i}}
\newcommand{\lineaj}{\variabile{j}}
\newcommand{\lineak}{\variabile{k}}
\newcommand{\nline}{\textrm{Nl}}
\newcommand{\n}{\variabile{n}}
\newcommand{\optangle}{\phi}
\newcommand{\psangle}{\tau}

\chapter*{List of symbols}


\begin{tabular}{l l}

$\textrm{d}{\Omega}$ & {Solid angle}\\
$\myangle$& {Angle between the direction of the solid angle and the normal $\boldsymbol{\nu}$}\\
$\myangle_i$& {Angle between the incident ray and the normal \mynormal}\\
$\myangle_r$ &{Angle between the reflected ray and the normal \mynormal}\\
$\myangle_t$ & {Angle between the transmitted ray and the normal \mynormal}\\
$\optangle$ & {Angle that the ray located on line \lineai forms with respect to the optical axis}\\

\end{tabular}

%\end{document}